\chapter{Lecture 12}
In this lecture, we continue with our discussion of reduced-order models which are a method of mapping models to models. In this lecture, we look at problems where we seek a reduced basis that represents the solution to our problem as well as projections into that basis. Examples of such projections include the singular-value decomposition, eigenbases, Krylov-subspace methods, and the Hankel transform.

\section{Example: Wavelet Transform}
Consider the linear system of equations
\begin{align} \label{eqn:12:linear}
\bs{L} \bs{u} = \bs{f}
\end{align}
Let $W$ be the matrix representing a wavelet transform. Since $W$ is unitary, we can rewrite
\begin{align} \label{eqn:12:waveletlinear}
\bs{W}\bs{L}\bs{W}^{*} \tilde{\bs{u}} &= \bs{\tilde{f}}
\end{align}
with $\tilde{\bs{u}} = \bs{W} \bs{u}$ and $\bs{\tilde{f}} = \bs{W}\bs{f}$. Since $\bs{\tilde{f}}$ is now in the wavelet basis, we can reshuffle its components so that its high-frequency components appear first. That is, we can use a unitary $\bs{Q}$ so that
\begin{align} \label{eqn:12:unitary}
\bs{Q} \bs{W} \bs{L} \bs{W}^{*} \bs{Q} \bs{\bar{u}} &= \bs{\bar{f}}
\end{align}
where $\bs{\bar{f}} = \bs{Q} \bs{\tilde{f}} = \bs{Q} \bs{W} \bs{f}$ and $\bs{\bar{u}} = \bs{Q} \bs{\tilde{u}} = \bs{Q} \bs{W} \bs{u}$. Finally, rewrite Equation \ref{eqn:12:unitary} as
\begin{align}
\begin{bmatrix}
\bs{A} & \bs{B} \\
\bs{C} & \bs{D}
\end{bmatrix} \begin{bmatrix}
    \bs{u}_{H} \\
    \bs{u}_{L}
\end{bmatrix} &= \begin{bmatrix}
\bs{f}_{H} \\
\bs{f}_{L}
\end{bmatrix} \\
\bs{u}_{H}, \bs{f}_{H} \in \R^{n_1} \\
\bs{u}_{L}, \bs{f}_{L} \in \R^{n_2}
\end{align}
with $n_1 \gg n_2$ and 
\begin{align}
    \begin{bmatrix}
    \bs{A} & \bs{B} \\
    \bs{C} & \bs{D}
    \end{bmatrix} = \bs{Q} \bs{W} \bs{L} \bs{W}^{*} \bs{Q}^{*}.
\end{align}
Here, we use the subscript $H$ to mean ``high frequency'' and $L$ to mean ``low frequency.'' Typically, noise lies in the high-frequency range, and so we are only interested in solving for $u_L$ and throwing away $u_H$, even though $dim(u_H) \gg dim(u_L)$!
From here, we can apply the same online/offline strategy outlined in the previous lecture. Again this approach is of practical use for many right hand sides $\bs{u}_L$.

\section{Example: Control Theory}
Consider Equation \ref{eqn:11:controlODE} from Lecture 11. We begin in the case when $f,g$ are linear. Then Equation \ref{eqn:11:controlODE} becomes
\begin{ceqn} \label{eqn:12:linearcontrol}
\dot{\bs{x}}(t) &= \bs{A} \bs{x}(t) + \bs{B} \bs{u}(t) \\
y(t) &= \bs{C} \bs{x}(t) + \bs{D} \bs{u}(t)
\end{ceqn}
with $dim(\bs{x}) \gg dim(\bs{y})$. Consider the case when $\bs{B} = \bs{D} = \bs{0}$ so that we have the further reduction
\begin{align} 
\dot{\bs{x}}(t) &= \bs{A} \bs{x}(t) \label{eqn:12:nou1} \\
\bs{y}(t) &= \bs{C} \bs{x}(t) \label{eqn:12:nou2}
\end{align}
We note that Equation \ref{eqn:12:nou2} leads to a drastic dimensionality reduction through the matrix $\bs{C}$. Is there any way that we can incorporate $\bs{C}$ into Equation \ref{eqn:12:nou1} to lead to an overall complexity reduction? We can see through straightforward algebra that
\begin{align} \label{eqn:12:reducey}
\dot{\bs{y}}(t) &= \bs{C} \bs{A} \bs{x}(t).
\end{align}
Note that the right-hand side of Equation \ref{eqn:12:reducey} does not have $\bs{y}$, so we do not see a direct dimensionality reduction from this equation, and for general $\bs{A}$, we cannot expect any reduction.   