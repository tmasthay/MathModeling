\chapter{Lecture 14}
We cover wellposedness in this lecture. As a review, we consider a problem well-posed whenever (a) solutions exist, (b) solutions are unique, and (c) solutions depend continuously on initial conditions.

\section{Wellposedness of ODEs}
Consider the simple ODE below.
\begin{ceqn} \label{eqn:14:ODE}
y'(t) &= f(y(t)) & t \in (0,T) \\
y(0) &= y_0.
\end{ceqn}
The wellposedness of this problem depends on the forcing function $f$. For example, whenever $f(y) = 2\sqrt{y}$, solutions exist but are not unique. If we take $y_0 = 0$, we see that $y(t) \equiv 0$ and $y(t) = t^2$ both solve Equation \eqref{eqn:14:ODE}. In general, $f$ Lipschitz is a sufficient condition for wellposedness and can be shown Picard iteration. Here, we outline the proof of showing continuous dependence on data. To prove this, we need Gronwall's inequality, stated below.
\begin{lemma}{(Gronwall's Inequality)} \label{thm:14:gronwall}
Let $u$ and $\beta$ be continuous on an interval $I$ which contains its infimum. Suppose $u$ is differentiable on $I^{0}$, the interior of $I$, and satisfies the differential inequality 
\begin{align} \label{eqn:gronwall}
u'(t) \leq \beta(t) u(t)
\end{align}
on $I^{0}$. Then for all $t \in I$,
\begin{align}
    u(t) \leq u(a) \int_{a}^{t} \beta(s) ds
\end{align}
\end{lemma}
\begin{theorem}
If $f$ is Lipschitz continuous, then solutions to Equation \eqref{eqn:14:ODE} depend continuously on the initial condition.
\end{theorem}
\begin{proof}
Let $y$ solve Equation \eqref{eqn:14:ODE} with initial condition $y(0) = y_0$ and let $u$ solve Equation \eqref{eqn:14:ODE} with initial condition $u(0) = y_0 - \epsilon$. Then for any $t \in (0,T)$, we have
\begin{ceqn} \label{eqn:14:derivbound}
    y'(t) - u'(t) &= f(y(t)) - f(u(t)) \\
    &\leq |f(y(t)) - f(u(t))| \\
    &\leq L |y(t) - u(t)|
\end{ceqn}
Note that $y(0) - u(0) = \epsilon$. Denoting $E(t) = y(t) - u(t)$, we can thus rewrite Equation \eqref{eqn:14:derivbound} as 
\begin{ceqn} \label{eqn:14:errinequal}
     E'(t) &\leq L |E(t)| \\
     E(0) &= \epsilon
\end{ceqn}
Applying Lemma \ref{thm:14:gronwall}, we have for any $t \in [0,T]$
\begin{ceqn}
E(t) &\leq \epsilon e^{L(t-a)} \\
&\leq M \epsilon
\end{ceqn}
where $M = e^{LT}$. Therefore, we have continuous dependence in $L^{\infty}$ and thus also all $L^{p}$ norms.
\end{proof}