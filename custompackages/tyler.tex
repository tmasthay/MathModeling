%Simple shortcut commands
\newcommand{\pint}[3]{\int_{#1} \left( #2 \right) d#3}
\newcommand{\sint}[3]{\int_{#1} #2 \hspace{3pt} d#3}
\newcommand{\bs}[1]{\boldsymbol{#1}}
\newcommand{\pd}[2]{\frac{\partial #1}{\partial #2}}
\newcommand{\mc}[1]{\mathcal{#1}}
\newcommand{\tu}[1]{\textup{#1}}
\newcommand{\mbb}[1]{\mathbb{#1}}
\newcommand{\argmin}[1]{\underset{#1}{\textup{arg min }}}
\newcommand{\tbf}[1]{\textbf{#1}}
\newcommand{\tbfr}[1]{\textbf{\textcolor{red}{#1}}}
\newcommand{\R}{\mathbb{R}}
\newcommand{\PR}{\mathbb{P}}
\newcommand{\N}{\mathbb{N}}
%\newcommand{\E}{\mathbb{E}}
\newcommand{\EE}[1]{\mathbb{E}\left[#1\right]}
\newcommand{\indprob}[1]{\mathds{1}_{\left\{#1\right\}}}
\newcommand{\inwedge}[1]{\langle #1 \rangle}
\newcommand{\ncc}[4]{\textcolor{#1}{#3}\textcolor{#2}{#4}}
\newcommand{\nccc}[6]{\ncc{#1}{#2}{#4}{#5}\textcolor{#3}{#6}}
\newcommand{\ncccc}[8]{\ncc{#1}{#2}{#5}{#6}\ncc{#3}{#4}{#7}{#8}}
\newcommand{\tc}[2]{\textcolor{#1}{#2}}
\newcommand{\tcr}[1]{\textcolor{red}{#1}}
\newcommand{\tcc}[1]{\texcolor{violet}{#1}}
\newcommand{\q}[1]{``#1''}
\newcommand{\bt}[1]{\textbf{#1}}
\newcommand{\txt}[1]{\textup{#1}}
\newcommand{\pref}[1]{(\ref{#1})}

\newcommand\subt[1]{%
  \pgfmathparse{#1-1}\pgfmathresult%
}

\newcommand{\cbeqn}[2]{
\begin{empheq}[box=\tcbhighmath]{gather}
 \textbf{#1} \nonumber \\
 #2
\end{empheq}
}

\newcommand{\bjorn}{Bj\"{o}rn}
\newcommand{\askbjorn}[1]{\textcolor{red}{Ask \bjorn: #1}}

%More complicated commands found on Internet
\newcommand*\widefbox[1]{\fbox{\hspace{2em}#1\hspace{2em}}}
\newcommand{\rpm}{\sbox0{$1$}\sbox2{$\scriptstyle\pm$}
  \raise\dimexpr(\ht0-\ht2)/2\relax\box2 }
 
%Custom environments
\newenvironment{ceqn}
    {\begin{equation}
    \begin{aligned}
    }
    { 
    \end{aligned} 
    \end{equation}
    }
    
